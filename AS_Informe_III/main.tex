\documentclass[10pt]{article}
\usepackage[spanish]{babel}
\usepackage[utf8]{inputenc}
\usepackage{graphicx}
\usepackage{hyperref}
\usepackage[lmargin=2cm, rmargin=2cm, top=2cm, bottom=2 cm]{geometry}
\usepackage{fancyhdr}
\pagestyle{fancy}
\usepackage{tabularx}

\usepackage[table,xcdraw]{xcolor}
\fancyhead{}
\fancyhead[R]{Facultad de Ingeniería y Ciencias \\ Instituto de Ciencias Básicas}
\fancyhead[L]{\includegraphics[width=3.2cm]{img/udp_logo.png}}
\usepackage{animate}
\usepackage{enumerate}
\usepackage{float}
\usepackage{url}
\usepackage{karnaugh-map}
\usepackage{longtable}

\fancyfoot{}
\fancyfoot[R]{Página \thepage \hspace{0.02 cm}}
\renewcommand{\headrulewidth}{0.9pt}
\renewcommand{\footrulewidth}{0.5pt}

\usepackage{listings}
\lstset{
  basicstyle=\ttfamily\small,
  breaklines=true,
  frame=single,
  columns=fullflexible
}

\usepackage{tabularx,booktabs,array}
\newcommand{\NA}{---}
\newcolumntype{Y}{>{\raggedright\arraybackslash}X} % última columna quebrable y alineada a la izquierda
\setlength{\extrarowheight}{2pt}

%New colors defined below
\definecolor{codegreen}{rgb}{0,0.6,0}
\definecolor{codegray}{rgb}{0.5,0.5,0.5}
\definecolor{codepurple}{rgb}{0.58,0,0.82}
\definecolor{backcolour}{rgb}{0.95,0.95,0.92}

%Code listing style named "mystyle"
\lstdefinestyle{mystyle}{
  backgroundcolor=\color{backcolour},   commentstyle=\color{codegreen},
  keywordstyle=\color{magenta},
  numberstyle=\tiny\color{codegray},
  stringstyle=\color{codepurple},
  basicstyle=\ttfamily\footnotesize,
  breakatwhitespace=false,
  breaklines=true,
  captionpos=b,
  keepspaces=true,
  numbers=left,
  numbersep=5pt,
  showspaces=false,
  showstringspaces=false,
  showtabs=false,
  tabsize=2
}
\lstset{style=mystyle}

\begin{document}
\date{}
\begin{titlepage}
\begin{center}
    \vspace*{\baselineskip}

    {
    \bf\fontsize{19}{0}{\selectfont{UNIVERSIDAD DIEGO PORTALES}}\\[0.5cm]
    \fontsize{11}{0}{FACULTAD DE INGENIERÍA Y CIENCIAS}
    }

    \vspace*{0.5\baselineskip}
    {
    \bf\fontsize{11}{0}{\selectfont{ESCUELA DE INFORMÁTICA Y TELECOMUNICACIONES}}\\[0.35cm]
    }

    \vspace*{\baselineskip}
    \includegraphics[scale=0.50]{img/udp_logo.png}
    \vspace*{3\baselineskip}
     \hrule height 0.5pt
    \vspace{1mm}
    \hrule height 1.5pt
    \vspace*{1\baselineskip}

    {
    \bf\fontsize{15}{0}{\selectfont{Arquitectura de software \\[0.3cm] Sistema de préstamo de materiales}}
    }
    \vspace*{1\baselineskip}
    \hrule height 0.5pt
    \vspace{1mm}
    \hrule height 1.5pt

    \vspace*{4.5\baselineskip}

    {
     \bf\fontsize{14}{0}{\selectfont{Profesor:\\[0.3cm]
}}

\bf\fontsize{14}{0}{\selectfont{Juan Ricardo Giadach Giadach\\[0.5cm]
}}
   \bf\fontsize{12}{0}{\selectfont{Estudiantes:\\[0.3cm]}}
   \bf\fontsize{12}{0}{\selectfont{Brayan Eduardo González Sánchez\\}}
   \bf\fontsize{12}{0}{\selectfont{Rafael Eduardo Campos Sepúlveda\\}}
   \bf\fontsize{12}{0}{\selectfont{Alejandro Ignacio Jara Vergara\\}}
   \bf\fontsize{12}{0}{\selectfont{Alexis Agustín Lema González\\}}
   }

    \vfill
    Santiago, Chile \hfill Agosto del 2025

\end{center}
\end{titlepage}
\vspace*{\baselineskip}
\tableofcontents
\setcounter{page}{0}
\thispagestyle{empty}
\newpage

\section{Descripción del sistema, la organización y el área}
    \subsection*{Descripción del sistema}
        \textbf{Nombre tentativo}: \emph{PrestaLab}.\\
        \emph{PrestaLab} es un sistema que permite \textbf{buscar}, \textbf{reservar} y \textbf{prestar} materiales con trazabilidad de usuarios y estados. Integra lo siguente:
        \begin{itemize}
            \item \textbf{Búsqueda por filtro} por nombre, tipo, sede y estado (disponible, prestado, dañado, perdido).
            \item \textbf{Reservas} con \emph{ventana de retiro} y caducidad automática si no se retira a tiempo.
            \item \textbf{Préstamo presencial} con registro de responsable y fecha de devolución.
            \item \textbf{Renovaciones} condicionadas a disponibilidad y máximos.
            \item \textbf{Multas} por atraso y bloqueo automático por deuda sobre umbral.
            \item \textbf{Sugerencias} para nuevos articulos. 
            \item \textbf{Gestión de daños/pérdidas} con generación de cargos.
            \item \textbf{Lista de espera} y \textbf{notificaciones} (recordatorios, atrasos, disponbilidad y avisos de caducidad).
            \item \textbf{Reportes de circulación} y \textbf{exportación} de historial (PDF/CSV) por usuario.
        \end{itemize}
    \subsection*{Organización y área}
        Trabajaremos con una universidad (facultad con biblioteca y laboratorios de docencia), donde circulan materiales como dispositivos electrónicos (tablets, notebooks, etc), libros, herramientas, kits de electrónica, instrumentos de laboratorio y accesorios (multímetros, fuentes, cables, etc.).
        Hoy la gestión se hace de forma mixta: planillas, correo y registros manuales en mesón. Eso genera problemas tales como: ítems que “desaparecen” del control, atrasos que nadie recuerda, materiales dañados que siguen en catálogo, colas en horas punta y poca visibilidad de qué está disponible y cuándo vuelve.
        El área beneficiada es la Biblioteca y el Laboratorio de la Facultad, en donde los usuarios principales son:
        \begin{itemize}
        \item \textbf{Estudiantes y docentes que piden y devuelven materiales.}
         \item \textbf{Encargado/a de biblioteca/lab que valida retiros y devoluciones, gestiona estado de ítems, multas y reportes.}
        \end{itemize}

\section{Objetivos del sistema y descripción de usuarios}
    \subsection*{Objetivos del sistema}
        \begin{enumerate}
            \item \textbf{Disponibilidad y trazabilidad}: Garantizar la disponibilidad y trazabilidad de los materiales para facilitar su localización y asegurar un control confiable de todas las operaciones.
            \item \textbf{Eficiencia operativa}: Optimizar la eficiencia operativa en la atención a usuarios, agilizando el proceso de préstamo y devolución.
            \item \textbf{Reducción de morosidad}: Disminuir los niveles de morosidad en los préstamos, fomentando el uso responsable de los recursos.
            \item \textbf{Integridad del inventario}: Preservar la integridad del inventario, garantizando la disponibilidad y el buen estado de los materiales.
            \item \textbf{Toma de decisiones}: Facilitar la toma de decisiones mediante reportes de uso que orienten la planificación de compras y mantenimiento.
        \end{enumerate}

    \subsection*{Usuarios del sistema}
        \begin{itemize}
            \item \textbf{Estudiante}: Busca, reserva, retira, devuelve, renueva (si procede), consulta su historial y descarga su registro (PDF/CSV).
            \item \textbf{Docente}: Mismas facultades que estudiante, con prioridad opcional en materiales docentes (si la política lo contempla).
            \item \textbf{Encargado/a (biblioteca/lab)}: Valida retiros y devoluciones, registra daños/pérdidas/robos, gestiona multas/bloqueos, administra lista de espera, emite reportes y parametriza políticas (ventanas, topes, umbrales).
        \end{itemize}

\section{Requerimientos funcionales}
    \subsection*{Listado de RF (Entrega 1)}
        \renewcommand{\arraystretch}{1.35}
        \setlength{\tabcolsep}{6pt}
        \begin{tabular}{|>{\raggedright\arraybackslash}p{1.2cm}|>{\raggedright\arraybackslash}p{3.5cm}|>{\raggedright\arraybackslash}p{9.2cm}|>{\raggedright\arraybackslash}p{2.2cm}|}
        \hline
        \textbf{ID} & \textbf{Nombre} & \textbf{Descripción} & \textbf{Usuario} \\ \hline
        RF01 & Búsqueda y filtros & Buscar ítems por Código identificador, nombre, tipo, ubicación y estado (disponible/prestado), mostrando cuando aplique la fecha estimada de devolución. & Estudiante / Docente \\ \hline
        RF02 & Reserva con ventana & Reservar un ítem con ventana de retiro; la reserva \textbf{caduca automáticamente} si no se retira a tiempo. & Estudiante / Docente \\ \hline
        RF03 & Préstamo presencial & Registrar el préstamo presencial con responsable y fecha de devolución pactada. & Encargado \\ \hline
        RF04 & Renovaciones & Permitir renovaciones automáticas si no hay reservas en cola y no se supera el máximo de renovaciones. & Estudiante / Docente \\ \hline
        RF05 & Multas por atraso & Calcular multa por atraso según tarifa por día & Todos \\ \hline
        RF06 & Bloqueo por deuda & Bloquear nuevos préstamos si existen multas impagas. & Encargado \\ \hline
        RF07 & Daños y pérdidas & Marcar ítem como dañado, perdido o robado y generar el cargo correspondiente, actualizando su estado. & Encargado \\ \hline
        RF08 & Lista de espera & Gestionar lista de espera por ítem; cuando se devuelve un articulo, se registra la operación, actualizando las existencias del articulo & Encargado \\ \hline
        RF09 & Sugerencias de articulos & Sugerir nuevos articulos o materiales o solicitar que aumenten las existencias de un articulo en particular. & Estudiante / Docente \\ \hline
        RF10 & Notificaciones & Enviar recordatorios y avisos a través de \textbf{correo}, \textbf{WhatsApp} y \textbf{portal (in-app)}, según las preferencias del usuario. Cubre devoluciones próximas, atrasos, disponibilidad y caducidad de reservas. & Encargado \\ \hline
        RF11 & Reportes de circulación & Entregar reportes de rotación por ítem, morosidad y daños/pérdidas/robo por periodo y sede. & Encargado \\ \hline
        RF12 & Exportar historial & Exportar un informe donde se detalle el historial de préstamos y reservas de un usuario (PDF/CSV). & Todos \\ \hline
        \end{tabular}

        \vspace{0.6cm}
        \newpage
        \subsection*{\textbf{Notas operativas:}}
        \begin{itemize}
            \item \textbf{RF01}: La fecha estimada de devolución, debe ser respecto al momento de hacer la consulta.
            
            \item \textbf{RF02:} Las reservas caducadas liberan el articulo; si hay registro en lista de espera (\textbf{RF08}) de este articulo, se notifica al usuario respectivo. Por otro lado si la reserva caduca (\textbf{RF02}), se le notifica al usuario que que su reserva caducó (\textbf{RF10}), además, el usuario no puede hacer reserva con ventana si tiene una multa por pagar.
            
            \item \textbf{RF04}: El usuario puede renovar el préstamo de un artículo, solamente si no existe un registro de este artículo en la lista de espera hecho por otro usuario.
            
            \item \textbf{Multas y bloqueos (RF05–RF06)} están orientados a reducir la morosidad y proteger la disponibilidad del inventario, el cambio de estado de un articulo debe afectar a las listas de espera si lo amerita.
            
            \item \textbf{RF07}: Existe un tope de dias de atraso, luego, se considera en estado "Robado".

            \item \textbf{RF10 – Notificaciones}: Por \textbf{portal}, \textbf{correo} o \textbf{WhatsApp} (según preferencias del usuario). Se envían ante eventos clave: reservas (creada, próxima a caducar, caducada), préstamos (confirmado), \emph{devolución próxima}, atraso/multa y disponibilidad para lista de espera.
            
            \item \textbf{RF12}: El informe de historial de prestamos y reservas de un usuario en particular, sólo lo puede solicitar el mismo usuario que consulta, o un encargado.
            
        \end{itemize}


\section{Mecanismo de persistencia de datos}

\subsection*{Modelo de datos}
El modelo se compone de las entidades principales:

\begin{figure}[H]
    \centering
    \includegraphics[width=1.05\linewidth]{img/prestalabdiagramabueno.png}
    \caption{Diagrama entidad-relación de prestalab}
\end{figure}

\subsection*{Diccionario de datos}

\paragraph{item}
\begingroup\footnotesize
\begin{tabularx}{\linewidth}{@{}l l c l l Y@{}}
\toprule
\textbf{Campo} & \textbf{Tipo} & \textbf{Nulo} & \textbf{PK/FK} & \textbf{Índice} & \textbf{Descripción / Dominio}\\
\midrule
id & BIGINT & No & PK & Sí (PK) & Identificador del ítem. (Índice compuesto adicional: \textit{itemIDX1}(nombre, tipo))\\
nombre & VARCHAR(50) & No & \NA &  & Nombre del material.\\
cantidad & INTEGER & No & \NA &  & Cantidad referencial en catálogo.\\
tipo & VARCHAR(20) & No & \NA &  & LIBRO, REVISTA, EQUIPO\_ELECTRONICO, OTRO.\\
valor & DECIMAL(10,2) & No & \NA &  & Valor referencial (CHECK valor \(\ge\) 0).\\
tarifa\_atraso & DECIMAL(10,2) & No & \NA &  & Tarifa diaria (CHECK tarifa\_atraso \(\ge\) 0).\\
descripcion & VARCHAR(100) & No & \NA &  & Descripción breve.\\
cantidad\_max & INTEGER & No & \NA &  & Tope por solicitud/préstamo (CHECK cantidad\_max \(>\) 0).\\
registro\_instante & DATETIME & No & \NA &  & Sello de creación.\\
\bottomrule
\end{tabularx}
\endgroup

\paragraph{sede}
\begingroup\footnotesize
\begin{tabularx}{\linewidth}{@{}l l c l l Y@{}}
\toprule
\textbf{Campo} & \textbf{Tipo} & \textbf{Nulo} & \textbf{PK/FK} & \textbf{Índice} & \textbf{Descripción / Dominio}\\
\midrule
id & BIGINT & No & PK & Sí (PK) & Identificador de la sede.\\
nombre & VARCHAR(50) & No & \NA &  & Nombre de la sede.\\
\bottomrule
\end{tabularx}
\endgroup

\paragraph{item\_existencia}
\begingroup\footnotesize
\begin{tabularx}{\linewidth}{@{}l l c l l Y@{}}
\toprule
\textbf{Campo} & \textbf{Tipo} & \textbf{Nulo} & \textbf{PK/FK} & \textbf{Índice} & \textbf{Descripción / Dominio}\\
\midrule
id & BIGINT & No & PK & Sí (PK) & Identificador de la copia física. (Índice \textit{item\_existenciaIDX1}(item\_id))\\
item\_id & BIGINT & No & FK \(\rightarrow\) \texttt{item.id} & Sí & Ítem al que pertenece.\\
sede\_id & BIGINT & No & FK \(\rightarrow\) \texttt{sede.id} & Sí & Sede de ubicación.\\
codigo & VARCHAR(50) & No & \NA & Único & Código único de la copia.\\
estado & VARCHAR(20) & No & \NA &  & DISPONIBLE, PRESTADO, MANTENIMIENTO, PERDIDO, DANNADO, ROBADO.\\
registro\_instante & DATETIME & No & \NA &  & Sello de creación.\\
\bottomrule
\end{tabularx}
\endgroup

\paragraph{usuario}
\begingroup\footnotesize
\begin{tabularx}{\linewidth}{@{}l l c l l Y@{}}
\toprule
\textbf{Campo} & \textbf{Tipo} & \textbf{Nulo} & \textbf{PK/FK} & \textbf{Índice} & \textbf{Descripción / Dominio}\\
\midrule
id & BIGINT & No & PK & Sí (PK) & Identificador de usuario. (Índice \textit{usuarioIDX1}(nombre))\\
nombre & VARCHAR(50) & No & \NA &  & Nombre completo.\\
correo & VARCHAR(50) & No & \NA & Único & Email (único).\\
tipo & VARCHAR(20) & No & \NA &  & ENCARGADO, ESTUDIANTE, DOCENTE.\\
telefono & VARCHAR(15) & No & \NA &  & Teléfono de contacto.\\
estado & VARCHAR(20) & No & \NA &  & ACTIVO, INACTIVO, SUSPENDIDO, DEUDOR, BLOQUEADO.\\
preferencias\_notificacion & INTEGER & No & \NA &  & Bitmask: PORTAL(1), WHATSAPP(2), EMAIL(4). Default 0.\\
registro\_instante & DATETIME & No & \NA &  & Sello de creación.\\
\bottomrule
\end{tabularx}
\endgroup

\paragraph{sugerencia}
\begingroup\footnotesize
\begin{tabularx}{\linewidth}{@{}l l c l l Y@{}}
\toprule
\textbf{Campo} & \textbf{Tipo} & \textbf{Nulo} & \textbf{PK/FK} & \textbf{Índice} & \textbf{Descripción / Dominio}\\
\midrule
id & BIGINT & No & PK & Sí (PK) & Identificador de la sugerencia.\\
usuario\_id & BIGINT & No & FK \(\rightarrow\) \texttt{usuario.id} & Sí & Usuario que sugiere.\\
sugerencia & VARCHAR(100) & No & \NA &  & Texto de la sugerencia.\\
estado & VARCHAR(20) & No & \NA &  & PENDIENTE, ACEPTADA, RECHAZADA.\\
registro\_instante & DATETIME & No & \NA &  & Sello de creación.\\
\bottomrule
\end{tabularx}
\endgroup

\paragraph{notificacion}
\begingroup\footnotesize
\begin{tabularx}{\linewidth}{@{}l l c l l Y@{}}
\toprule
\textbf{Campo} & \textbf{Tipo} & \textbf{Nulo} & \textbf{PK/FK} & \textbf{Índice} & \textbf{Descripción / Dominio}\\
\midrule
id & BIGINT & No & PK & Sí (PK) & Identificador de la notificación. (Índice \textit{notificacionIDX1}(usuario\_id))\\
usuario\_id & BIGINT & No & FK \(\rightarrow\) \texttt{usuario.id} & Sí & Destinatario.\\
canal & VARCHAR(20) & No & \NA &  & EMAIL, WHATSAPP, PORTAL.\\
tipo & VARCHAR(20) & No & \NA &  & RECORDATORIO, LISTA\_ESPERA, SUGERENCIA\_ACEPTADA, SUGERENCIA\_RECHAZADA, ATRASO, RESERVA\_CREADA, RESERVA\_PROXIMA\_CADUCAR, RESERVA\_CADUCADA, PRESTAMO\_CONFIRMADO, DEVOLUCION\_PROXIMA.\\
mensaje & TEXT & No & \NA &  & Contenido enviado.\\
registro\_instante & DATETIME & No & \NA &  & Sello de creación.\\
\bottomrule
\end{tabularx}
\endgroup

\paragraph{solicitud}
\begingroup\footnotesize
\begin{tabularx}{\linewidth}{@{}l l c l l Y@{}}
\toprule
\textbf{Campo} & \textbf{Tipo} & \textbf{Nulo} & \textbf{PK/FK} & \textbf{Índice} & \textbf{Descripción / Dominio}\\
\midrule
id & BIGINT & No & PK & Sí (PK) & Identificador de la solicitud. (Índice \textit{solicitudIDX1}(usuario\_id))\\
usuario\_id & BIGINT & No & FK \(\rightarrow\) \texttt{usuario.id} & Sí & Solicitante.\\
tipo & VARCHAR(30) & No & \NA &  & VENTANA, PRESTAMO, RENOVACION.\\
estado & VARCHAR(20) & No & \NA &  & PENDIENTE, ACEPTADA, RECHAZADA.\\
registro\_instante & DATETIME & No & \NA &  & Sello de creación.\\
\bottomrule
\end{tabularx}
\endgroup

\paragraph{item\_solicitud}
\begingroup\footnotesize
\begin{tabularx}{\linewidth}{@{}l l c l l Y@{}}
\toprule
\textbf{Campo} & \textbf{Tipo} & \textbf{Nulo} & \textbf{PK/FK} & \textbf{Índice} & \textbf{Descripción / Dominio}\\
\midrule
solicitud\_id & BIGINT & No & PK (comp.), FK \(\rightarrow\) \texttt{solicitud.id} & Sí & Cabecera de la solicitud.\\
item\_id & BIGINT & No & PK (comp.), FK \(\rightarrow\) \texttt{item.id} & Sí & Ítem solicitado.\\
cantidad & INTEGER & No & \NA &  & Cantidad (CHECK cantidad \(>\) 0).\\
registro\_instante & DATETIME & No & \NA &  & Sello de creación. (Índice \textit{item\_solicitudIDX1}(item\_id))\\
\bottomrule
\end{tabularx}
\endgroup

\paragraph{prestamo}
\begingroup\footnotesize
\begin{tabularx}{\linewidth}{@{}l l c l l Y@{}}
\toprule
\textbf{Campo} & \textbf{Tipo} & \textbf{Nulo} & \textbf{PK/FK} & \textbf{Índice} & \textbf{Descripción / Dominio}\\
\midrule
id & BIGINT & No & PK & Sí (PK) & Identificador del préstamo. (Índice \textit{prestamoIDX1}(item\_existencia\_id))\\
item\_existencia\_id & BIGINT & No & FK \(\rightarrow\) \texttt{item\_existencia.id} & Sí & Copia prestada.\\
solicitud\_id & BIGINT & No & FK \(\rightarrow\) \texttt{solicitud.id} & Sí & Origen del préstamo.\\
fecha\_prestamo & DATETIME & No & \NA &  & Inicio del préstamo.\\
fecha\_devolucion & DATETIME & Sí & \NA &  & Fecha límite pactada o real (si devuelto).\\
comentario & TEXT & Sí & \NA &  & Observaciones.\\
estado & VARCHAR(20) & No & \NA &  & ACTIVO, DEVUELTO, VENCIDO, PERDIDO.\\
renovaciones\_realizadas & INTEGER & No & \NA &  & Conteo de renovaciones (default 0).\\
registro\_instante & DATETIME & No & \NA &  & Sello de creación.\\
\bottomrule
\end{tabularx}
\endgroup

\paragraph{atraso}
\begingroup\footnotesize
\begin{tabularx}{\linewidth}{@{}l l c l l Y@{}}
\toprule
\textbf{Campo} & \textbf{Tipo} & \textbf{Nulo} & \textbf{PK/FK} & \textbf{Índice} & \textbf{Descripción / Dominio}\\
\midrule
id & BIGINT & No & PK & Sí (PK) & Identificador del atraso. (Índice \textit{atrasoIDX1}(prestamo\_id))\\
prestamo\_id & BIGINT & No & FK \(\rightarrow\) \texttt{prestamo.id} & Sí & Préstamo asociado.\\
dias\_atraso & INTEGER & No & \NA &  & Días de atraso (CHECK dias\_atraso \(>\) 0).\\
estado & VARCHAR(20) & No & \NA &  & PENDIENTE, NOTIFICADO, PAGADO, NO\_PAGADO.\\
registro\_instante & DATETIME & No & \NA &  & Sello de creación.\\
\bottomrule
\end{tabularx}
\endgroup

\paragraph{multa}
\begingroup\footnotesize
\begin{tabularx}{\linewidth}{@{}l l c l l Y@{}}
\toprule
\textbf{Campo} & \textbf{Tipo} & \textbf{Nulo} & \textbf{PK/FK} & \textbf{Índice} & \textbf{Descripción / Dominio}\\
\midrule
id & BIGINT & No & PK & Sí (PK) & Identificador de la multa. (Índice \textit{multaIDX1}(prestamo\_id))\\
prestamo\_id & BIGINT & No & FK \(\rightarrow\) \texttt{prestamo.id} & Sí & Préstamo asociado.\\
motivo & VARCHAR(20) & No & \NA &  & ATRASO, DANNOS, ROBO, PERDIDA.\\
valor & DECIMAL(10,2) & No & \NA &  & Monto (CHECK valor \(>\) 0).\\
estado & VARCHAR(20) & No & \NA &  & PENDIENTE, NOTIFICADO, PAGADO, NO\_PAGADO.\\
registro\_instante & DATETIME & No & \NA &  & Sello de creación.\\
\bottomrule
\end{tabularx}
\endgroup

\paragraph{ventana}
\begingroup\footnotesize
\begin{tabularx}{\linewidth}{@{}l l c l l Y@{}}
\toprule
\textbf{Campo} & \textbf{Tipo} & \textbf{Nulo} & \textbf{PK/FK} & \textbf{Índice} & \textbf{Descripción / Dominio}\\
\midrule
id & BIGINT & No & PK & Sí (PK) & Identificador de la ventana de retiro.\\
solicitud\_id & BIGINT & No & FK \(\rightarrow\) \texttt{solicitud.id} & Sí & Solicitud que origina la ventana.\\
item\_existencia\_id & BIGINT & No & FK \(\rightarrow\) \texttt{item\_existencia.id} & Sí & Copia reservada.\\
inicio & DATETIME & No & \NA &  & Inicio de la ventana.\\
fin & DATETIME & No & \NA &  & Fin de la ventana (CHECK fin \(>\) inicio).\\
\bottomrule
\end{tabularx}
\endgroup

\paragraph{lista\_espera}
\begingroup\footnotesize
\begin{tabularx}{\linewidth}{@{}l l c l l Y@{}}
\toprule
\textbf{Campo} & \textbf{Tipo} & \textbf{Nulo} & \textbf{PK/FK} & \textbf{Índice} & \textbf{Descripción / Dominio}\\
\midrule
id & BIGINT & No & PK & Sí (PK) & Identificador del registro en la lista. (Índice \textit{lista\_esperaIDX3}(fecha\_ingreso))\\
solicitud\_id & BIGINT & No & FK \(\rightarrow\) \texttt{solicitud.id} & Sí & Solicitud del usuario.\\
item\_id & BIGINT & No & FK \(\rightarrow\) \texttt{item.id} & Sí & Ítem esperado.\\
fecha\_ingreso & DATETIME & No & \NA &  & Marca temporal de ingreso.\\
estado & VARCHAR(20) & No & \NA &  & Estado (dominio no fijado en SQL).\\
registro\_instante & DATETIME & No & \NA &  & Sello de creación.\\
\bottomrule
\end{tabularx}
\endgroup

\paragraph{configuracion\_sistema}
\begingroup\footnotesize
\begin{tabularx}{\linewidth}{@{}l l c l l Y@{}}
\toprule
\textbf{Campo} & \textbf{Tipo} & \textbf{Nulo} & \textbf{PK/FK} & \textbf{Índice} & \textbf{Descripción / Dominio}\\
\midrule
id & BIGINT & No & PK & Sí (PK) & Identificador del parámetro.\\
clave & VARCHAR(50) & No & \NA & Único & Llave de configuración (única).\\
valor & VARCHAR(100) & No & \NA &  & Valor asociado.\\
descripcion & TEXT & Sí & \NA &  & Descripción del parámetro.\\
registro\_instante & DATETIME & No & \NA &  & Sello de creación.\\
\bottomrule
\end{tabularx}
\endgroup

\paragraph*{Lógica y uso de configuracion\_sistema}
\begingroup\footnotesize
\begin{itemize}
  \item \textbf{Propósito}: almacén centralizado clave–valor para parámetros de negocio y operación; la clave es única y el valor se interpreta según la clave (entero, decimal, horas/días), evitando cambios de esquema.
  \item \textbf{Lectura en tiempo de ejecución}: los servicios consultan estos parámetros al aplicar reglas (caducidad de reservas, topes de renovación, multas).
  \item \textbf{Integración por servicio} (ejemplos):
    \begin{itemize}
      \item \textbf{prart}: horas\_expiracion\_reserva, max\_renovaciones\_usuario, max\_items\_por\_usuario.
      \item \textbf{multa}: tarifa\_base\_multa, valor\_multa\_por\_dia\_atraso, porcentaje\_multa\_danos, porcentaje\_multa\_perdida, porcentaje\_multa\_robo, dias\_gracia\_devolucion.
      \item \textbf{notis / lista}: dias\_notificacion\_devolucion\_proxima, dias\_notificacion\_reserva\_proxima\_caducar, max\_items\_reserva\_simultanea.
      \item \textbf{regist}: tiempo\_bloqueo\_usuario\_deudor\_dias, notificar\_atraso\_cada\_dias.
    \end{itemize}
  \item \textbf{Validación sugerida}: castear el valor al tipo esperado por clave, validar rangos (\(\ge 0\)) y registrar quién/cuándo modifica (auditoría en la capa de aplicación).
  \item \textbf{Ejemplos de claves/valores}: max\_renovaciones\_usuario=3; horas\_expiracion\_reserva=24; valor\_multa\_por\_dia\_atraso=1000.00; dias\_notificacion\_devolucion\_proxima=3; tiempo\_bloqueo\_usuario\_deudor\_dias=30.
\end{itemize}
\endgroup


\section{Estructuración de funcionalidades}
Dado que se definió que el sistema \textbf{PrestaLab} adoptará una arquitectura \textbf{SOA (Service Oriented Architecture)}, las funcionalidades se han organizado en \textbf{componentes cliente} y \textbf{componentes servicio}, con responsabilidades bien delimitadas.

\subsection{Funcionalidades de los componentes de cliente}
\begin{itemize}
  \item \textbf{Portal web (estudiantes y docentes):}
  \begin{itemize}
    \item Búsqueda y reserva de materiales (RF01, RF02).
    \item Solicitud de renovación (si procede) y visualización de estado de préstamos (RF04).
    \item Consulta y exportación de su propio historial (RF12).
    \item Envío y consulta de sugerencias de artículos (RF09).
    \item Gestión de preferencias y recepción de notificaciones in-app (RF10).
  \end{itemize}

  \item \textbf{Módulo de administración (encargado/a de biblioteca o laboratorio):}
  \begin{itemize}
    \item Registro de préstamos y de devoluciones presenciales (RF03).
    \item Gestión de daños, pérdidas o robo de existencias (RF07).
    \item Administración de listas de espera (altas/bajas/atención) (RF08).
    \item Gestión de multas y bloqueos (RF05, RF06).
    \item Generación de reportes y \emph{exportación} de historiales por usuario (RF11, RF12).
    \item (Opcional) Administración de parámetros operativos del sistema (\emph{configuracion\_sistema}).
  \end{itemize}
\end{itemize}

\subsection{Funcionalidades de los componentes de servicio (SOA)}
\begin{itemize}
  \item \textbf{prart (Préstamos \& Artículos)}: catálogo, reservas con ventana, préstamos, devoluciones, renovaciones y marcación de estado de existencias (RF01, RF02, RF03, RF04, RF07).
  \item \textbf{lista (Listas de espera)}: altas/bajas y turnos por ítem, notificación al siguiente cuando queda disponible (RF08).
  \item \textbf{notis (Notificaciones)}: envíos multicanal según eventos del dominio (RF10).
  \item \textbf{multa (Penalizaciones)}: cálculo/registro de multas y aplicación de bloqueos por deuda (RF05, RF06).
  \item \textbf{sugit (Sugerencias)}: recepción y gestión de sugerencias de nuevos ítems o aumento de stock (RF09).
  \item \textbf{gerep (Reportes)}: reportes de circulación y exportación de historiales (RF11, RF12).
  \item \textbf{regist (Cuentas y roles)}: registro, autenticación y control de acceso \emph{(apoyo transversal a todos los RF)}.
\end{itemize}

\section{Interfaces de los componentes}
Cada componente expone una interfaz que vincula las operaciones con los \textbf{requerimientos funcionales (RF)}. Los \textbf{requerimientos no funcionales (RNF)} se dejan \emph{por definir} para completar posteriormente.

\subsection*{Componentes de cliente}

\subsubsection*{Portal web / aplicación móvil (estudiantes y docentes)}
\begin{itemize}
  \item \textbf{Operaciones principales (invocan servicios)}:
  \begin{itemize}
    \item \texttt{GET /items?filtros} $\rightarrow$ buscar ítems (RF01).
    \item \texttt{POST /reservas} $\rightarrow$ crear reserva con ventana (RF02).
    \item \texttt{DELETE /reservas/\{id\}} $\rightarrow$ cancelar reserva propia (RF02).
    \item \texttt{PUT /prestamos/\{id\}/renovar} $\rightarrow$ solicitar renovación (RF04).
    \item \texttt{GET /mi/historial?formato=pdf|csv} $\rightarrow$ exportar historial propio (RF12).
    \item \texttt{POST /sugerencias} $\rightarrow$ enviar sugerencia (RF09).
    \item \texttt{GET /notificaciones} y \texttt{GET/PUT /preferencias} $\rightarrow$ ver/gestionar preferencias (RF10).
  \end{itemize}
  \item \textbf{RF cubiertos por el cliente}: RF01, RF02, RF04, RF09, RF10, RF12.
\end{itemize}

\subsubsection*{Módulo de administración (encargado/a)}
\begin{itemize}
  \item \textbf{Operaciones principales (invocan servicios)}:
  \begin{itemize}
    \item \texttt{POST /prestamos} $\rightarrow$ registrar préstamo presencial (RF03).
    \item \texttt{POST /devoluciones} $\rightarrow$ registrar devolución presencial (RF03).
    \item \texttt{PUT /items/\{existenciaId\}/estado} $\rightarrow$ marcar dañado/perdido/robado (RF07).
    \item \texttt{GET /lista-espera/\{item\}} $\rightarrow$ consultar cola (RF08).
    \item \texttt{POST /lista-espera} / \texttt{DELETE /lista-espera/\{id\}} $\rightarrow$ altas/bajas (RF08).
    \item \texttt{GET /usuarios/\{id\}/multas} y \texttt{PUT /usuarios/\{id\}/bloqueo} $\rightarrow$ multas/bloqueos (RF05, RF06).
    \item \texttt{GET /reportes/circulacion} y \texttt{GET /usuarios/\{id\}/historial?formato=...} $\rightarrow$ reportes e historiales (RF11, RF12).
    \item \texttt{GET/PUT /configuracion} $\rightarrow$ (opcional) gestionar parámetros operativos.
  \end{itemize}
  \item \textbf{RF cubiertos por el cliente}: RF03, RF05, RF06, RF07, RF08, RF11, RF12.
\end{itemize}

\subsection*{Componentes de servicio (SOA)}

\subsubsection*{Servicio \textbf{prart} (Préstamos \& Artículos)}
\begin{itemize}
  \item \textbf{Operaciones (API)}:
  \begin{itemize}
    \item \texttt{GET /items?nombre|tipo|sede|estado} $\rightarrow$ búsqueda de ítems (RF01).
    \item \texttt{POST /reservas} $\rightarrow$ crear reserva con ventana (RF02).
    \item \texttt{DELETE /reservas/\{id\}} $\rightarrow$ cancelar/caducar reserva (RF02).
    \item \texttt{POST /prestamos} $\rightarrow$ registrar préstamo (RF03).
    \item \texttt{POST /devoluciones} $\rightarrow$ registrar devolución (RF03).
    \item \texttt{PUT /prestamos/\{id\}/renovar} $\rightarrow$ renovar préstamo (RF04).
    \item \texttt{PUT /items/\{existenciaId\}/estado} $\rightarrow$ actualizar estado (dañado/perdido/robado) (RF07).
  \end{itemize}
  \item \textbf{RF cubiertos}: RF01, RF02, RF03, RF04, RF07.
\end{itemize}

\subsubsection*{Servicio \textbf{lista} (Listas de espera)}
\begin{itemize}
  \item \textbf{Operaciones (API)}:
  \begin{itemize}
    \item \texttt{POST /lista-espera} $\rightarrow$ agregar usuario a la cola por ítem (RF08).
    \item \texttt{DELETE /lista-espera/\{id\}} $\rightarrow$ eliminar usuario de la cola (RF08).
    \item \texttt{GET /lista-espera/\{item\}} $\rightarrow$ consultar cola de un ítem (RF08).
    \item \texttt{POST /lista-espera/\{id\}/notificar-siguiente} $\rightarrow$ notificar disponibilidad (RF08).
  \end{itemize}
  \item \textbf{RF cubiertos}: RF08.
  \item \textbf{RNF asociados}:
  \begin{itemize}
    \item \textbf{Equidad}: mantener orden \emph{FIFO}. En condiciones de concurrencia, ningún usuario puede ser adelantado por más de \textbf{1 posición}.
    \item \textbf{Notificación}: al registrarse una devolución que libera un ítem, publicar el evento de disponibilidad en \textbf{menos de 5 segundos}.
  \end{itemize}
\end{itemize}

\subsubsection*{Servicio \textbf{notis} (Notificaciones)}
\begin{itemize}
  \item \textbf{Operaciones (API)}:
  \begin{itemize}
    \item \texttt{POST /notificaciones} $\rightarrow$ enviar aviso (RF10).
    \item \texttt{GET /preferencias/\{usuarioId\}} y \texttt{PUT /preferencias/\{usuarioId\}} $\rightarrow$ consultar/actualizar preferencias (RF10).
    \item \emph{Suscripción a eventos de dominio}: ReservaCreada, ReservaProximaACaducar, ReservaCaducada, PrestamoConfirmado, DevolucionProxima, Atraso.
  \end{itemize}
  \item \textbf{RF cubiertos}: RF10.
  \item \textbf{RNF asociados}:
  \begin{itemize}
    \item \textbf{Latencia de entrega}: enviar las notificaciones \textbf{dentro de 5 minutos} desde que ocurre el evento.
  \end{itemize}
\end{itemize}

\subsubsection*{Servicio \textbf{multa} (Penalizaciones)}
\begin{itemize}
  \item \textbf{Operaciones (API)}:
  \begin{itemize}
    \item \texttt{GET /usuarios/\{id\}/multas} $\rightarrow$ consultar deudas/multas (RF05).
    \item \texttt{POST /multas} $\rightarrow$ registrar multa (RF05).
    \item \texttt{PUT /usuarios/\{id\}/bloqueo} $\rightarrow$ aplicar/quitar bloqueo por deuda (RF06).
  \end{itemize}
  \item \textbf{RF cubiertos}: RF05, RF06.
\end{itemize}

\subsubsection*{Servicio \textbf{sugit} (Sugerencias)}
\begin{itemize}
  \item \textbf{Operaciones (API)}:
  \begin{itemize}
    \item \texttt{POST /sugerencias} $\rightarrow$ enviar sugerencia (RF09).
    \item \texttt{GET /sugerencias} $\rightarrow$ listar sugerencias (RF09).
    \item \texttt{PUT /sugerencias/\{id\}/aprobar} \,|\, \texttt{PUT /sugerencias/\{id\}/rechazar} $\rightarrow$ gestión (RF09).
  \end{itemize}
  \item \textbf{RF cubiertos}: RF09.
\end{itemize}

\subsubsection*{Servicio \textbf{regist} (Cuentas y roles)}
\begin{itemize}
  \item \textbf{Operaciones (API)}:
  \begin{itemize}
    \item \texttt{POST /usuarios} $\rightarrow$ registrar usuario.
    \item \texttt{POST /auth/login} $\rightarrow$ autenticar usuario.
    \item \texttt{PUT /usuarios/\{id\}/roles} $\rightarrow$ asignar/actualizar roles y permisos.
    \item \texttt{GET /usuarios/\{id\}} $\rightarrow$ consultar perfil; \texttt{PUT /usuarios/\{id\}} $\rightarrow$ actualizar datos.
  \end{itemize}
  \item \textbf{RF cubiertos}: \emph{Apoyo transversal a RF01–RF12}.
\end{itemize}

\subsubsection*{Servicio \textbf{gerep} (Reportes)}
\begin{itemize}
  \item \textbf{Operaciones (API)}:
  \begin{itemize}
    \item \texttt{GET /reportes/circulacion?periodo\&sede} $\rightarrow$ métricas de rotación, morosidad y daños (RF11).
    \item \texttt{GET /usuarios/\{id\}/historial?formato=pdf|csv} $\rightarrow$ exportar historial individual (RF12).
  \end{itemize}
  \item \textbf{RF cubiertos}: RF11, RF12.
\end{itemize}

\section{Documentación del sistema (entrega 3)}
    \subsection{Componentes de servicio}

    El sistema \textbf{PRESTALAB} está diseñado bajo una arquitectura orientada a servicios (\textbf{SOA}), compuesta por múltiples servicios independientes que gestionan distintos dominios funcionales del proceso de préstamos, usuarios y notificaciones.

    Cada servicio expone su API mediante \textbf{endpoints RESTful} que se comunican a través del protocolo HTTP, utilizando estructuras JSON para entrada y salida de datos.
    
    \subsubsection{Resumen general de servicios}
        \begin{longtable}{@{}llll@{}}
        \toprule
        \textbf{Dominio} & \textbf{Servicio} & \textbf{Puerto} & \textbf{Función principal} \\ \midrule
        Reportes & GEREP & 8001 & Generación de reportes históricos. \\
        Lista de Espera & LISTA & 8002 & Gestión de solicitudes en espera. \\
        Multas & MULTA & 8003 & Administración de sanciones. \\
        Notificaciones & NOTIS & 8004 & Comunicación y envío de notificaciones. \\
        Préstamos & PRART & 8005 & Administración de préstamos, reservas y devoluciones. \\
        Usuarios & REGIST & 8006 & Registro y autenticación de usuarios. \\
        Sugerencias & SUGIT & 8007 & Gestión de sugerencias o feedback. \\ \bottomrule
        \end{longtable}
    
    \subsubsection{Detalle de servicios}
    
        \paragraph{1. Servicio de Reportes (GEREP)} \mbox{}\\
        \textbf{Puerto:} 8001 \\
        \textbf{Responsabilidad:} Generar reportes de historial de préstamos por usuario.
        
        \begin{longtable}{@{}lll@{}}
            \toprule
            \textbf{Método} & \textbf{Ruta} & \textbf{Descripción} \\ 
            \midrule
            GET & /usuarios/\{usuario\_id\}/historial?formato=csv & 
            Devuelve el historial de préstamos\\
            \bottomrule
        \end{longtable}
        
        \paragraph{2. Servicio de Lista de Espera (LISTA)} \mbox{}\\
        \textbf{Puerto:} 8002 \\
        \textbf{Responsabilidad:} Gestionar solicitudes en lista de espera para ítems no disponibles.
        
        \begin{longtable}{@{}lll@{}}
        \toprule
        \textbf{Método} & \textbf{Ruta} & \textbf{Descripción} \\ \midrule
        POST & /lista-espera/ & Registrar una nueva solicitud. \\
        PUT & /lista-espera/\{id\} & Actualizar el estado de una solicitud. \\
        GET & /lista-espera/\{item\_id\} & Obtener registros por ID de ítem. \\ \bottomrule
        \end{longtable}
        
        \paragraph{3. Servicio de Multas (MULTA)} \mbox{}\\
        \textbf{Puerto:} 8003 \\
        \textbf{Responsabilidad:} Registrar y administrar multas por atrasos u otros motivos.
        
        \begin{longtable}{@{}lll@{}}
        \toprule
        \textbf{Método} & \textbf{Ruta} & \textbf{Descripción} \\ \midrule
        POST & /multas & Registrar una multa asociada a un préstamo. \\
        GET & /usuarios/\{usuario\_id\}/multas & Obtener todas las multas de un usuario. \\
        PUT & /usuarios/\{usuario\_id\}/estado & Cambiar el estado del usuario. \\ \bottomrule
        \end{longtable}
        
        \paragraph{4. Servicio de Notificaciones (NOTIS)} \mbox{}\\
        \textbf{Puerto:} 8004 \\
        \textbf{Responsabilidad:} Enviar notificaciones y administrar preferencias.
        
        \begin{longtable}{@{}lll@{}}
        \toprule
        \textbf{Método} & \textbf{Ruta} & \textbf{Descripción} \\ \midrule
        POST & /notificaciones & Enviar una notificación a un usuario. \\
        GET & /preferencias/\{usuario\_id\} & Obtener preferencias de notificación. \\
        PUT & /preferencias/\{usuario\_id\} & Actualizar preferencias del usuario. \\ \bottomrule
        \end{longtable}
        
        \paragraph{5. Servicio de Préstamos y Artículos (PRART)} \mbox{}\\
        \textbf{Puerto:} 8005 \\
        \textbf{Responsabilidad:} Gestionar el ciclo de vida de los préstamos, reservas y existencias.
        
        \begin{longtable}{@{}lll@{}}
        \toprule
        \textbf{Método} & \textbf{Ruta} & \textbf{Descripción} \\ \midrule
        GET & /items?tipo=\&nombre= & Consultar ítems filtrados. \\
        POST & /reservas & Crear una ventana de reserva. \\
        DELETE & /reservas/\{id\} & Eliminar una reserva existente. \\
        POST & /solicitudes & Registrar solicitud de préstamo o renovación. \\
        POST & /prestamos & Registrar un nuevo préstamo. \\
        PUT & /prestamos/\{id\}/renovar & Renovar un préstamo existente. \\
        POST & /devoluciones & Registrar la devolución de un préstamo. \\
        PUT & /items/\{id\}/estado & Actualizar estado de un ítem. \\ \bottomrule
        \end{longtable}
        
        \paragraph{6. Servicio de Registro y Usuarios (REGIST)} \mbox{}\\
        \textbf{Puerto:} 8006 \\
        \textbf{Responsabilidad:} Autenticación y gestión de usuarios.
        
        \begin{longtable}{@{}lll@{}}
        \toprule
        \textbf{Método} & \textbf{Ruta} & \textbf{Descripción} \\ \midrule
        POST & /usuarios & Registrar un nuevo usuario. \\
        POST & /auth/login & Autenticar usuario. \\
        GET & /usuarios/\{id\} & Obtener información de usuario. \\
        PUT & /usuarios/\{id\} & Actualizar información o preferencias. \\ \bottomrule
        \end{longtable}
        
        \paragraph{7. Servicio de Sugerencias (SUGIT)} \mbox{}\\
        \textbf{Puerto:} 8007 \\
        \textbf{Responsabilidad:} Registrar y administrar sugerencias.
        
        \begin{longtable}{@{}lll@{}}
        \toprule
        \textbf{Método} & \textbf{Ruta} & \textbf{Descripción} \\ \midrule
        POST & /sugerencias & Registrar una nueva sugerencia. \\
        GET & /sugerencias & Obtener todas las sugerencias. \\
        PUT & /sugerencias/\{id\}/aprobar & Aprobar una sugerencia. \\
        PUT & /sugerencias/\{id\}/rechazar & Rechazar una sugerencia. \\ \bottomrule
        \end{longtable}
    \subsection{Componentes de cliente}
    \subsection{Componente bus: ESB}
\section{Análisis crítico del sistema (entrega 3)}
    \subsection{Problemas identificados}
        \subsubsection{Componentes de servicio}
            En el diseño, desarrollo y despliegue se evidenció un problema desde el comienzo: los servicios pueden no comunicarse correctamente entre sí y la BBDD, debiéndose a configuraciones complejas en las variables de entorno. 
        \subsubsection{Componentes de cliente}
        \subsubsection{Componente bus: ESB}
        \subsubsection{Otros}
    \subsection{Soluciones desplegadas}
        \subsubsection{Componentes de servicio}
            Para mitigar el eecto del problema de las variables de entorno, se decidió centralizar y normalizar las variables de entorno de los servicios en un ficher .env
        \subsubsection{Componentes de cliente}
        \subsubsection{Componente bus: ESB}
        \subsubsection{Otros}
\section{Anexo}
  \href{https://github.com/alexisnasus/prestalab_SOA/blob/main}{Repositorio GitHub} 
\end{document}
